\documentclass[11pt,a4paper,sans]{moderncv}
\moderncvstyle{casual} 
\moderncvcolor{blue}
\setlength{\hintscolumnwidth}{2cm} 
\usepackage[utf8]{inputenc}
\usepackage[scale=0.9]{geometry}
\usepackage{helvet}
\usepackage[french]{babel}
\name{Mickaël}{Moreau}
\title{Ingénieur en micro-électronique} 
\address{50 lotissement Rivoire de la Dame}{38360 Sassenage, France}
\phone[mobile]{06~88~33~15~88}
%\phone[fixed]{01~01~88~33~55}
%\phone[fax]{02~11~22~33~44}
\email{mickael.moreau@gmail.com}
%\homepage{www.pierredurand.com}
%\social[linkedin]{pierre.durand}
%\social[twitter]{pierre.durand}
%\social[github]{pierre.durand}
\extrainfo{41 ans - Marié - 3 enfants}
%\photo[64pt][0.4pt]{maphoto}
%\quote{Ingénieur en micro-électronique}
\begin{document}
\makecvtitle
\section{Experience professionnelle}
\cventry{2014}{Ingénieur vérification d’IP graphique}{STMicroelectronics}{Grenoble}{}%
{%
\begin{itemize}%
\item Modernisation de l'environnement (ex : Git, Jenkins,..).
\item Développement des LLD (Low-Level Driver)
\end{itemize}}

\cventry{2008--2014}{Ingénieur vérification en centrale}{STMicroelectronics}{Grenoble}{}%
{%
\begin{itemize}%
\item Développement, évaluation et déploiement des outils de vérification :
 \begin{itemize}%
 \item \textit{Verification-Cockpit} : environnement de vérification générique.
\item Base de données MySQL pour enregistrer l'historique de la vérification.
\item Rapport pour suivre l'évolution de la vérification.
\item Environnement de vérification pour la nouvelle génération de SOC.
\end{itemize}
\item Prise en charge des benchmarks des simulateurs (RTL et Gates).
\end{itemize}}

\cventry{2006--2008}{Ingénieur CAD support}{STMicroelectronics}{Grenoble}{}%
{%
\begin{itemize}%
\item Installation des logiciels et des licences.
\item Support des outils de vérifications et de simulations.
\item Développement et support des outils de navigation.
\end{itemize}}

\cventry{1999--2006}{Ingénieur en analyse de défaillance électrique}{STMicroelectronics}{Grenoble}{}{%
\begin{itemize}
\item Développement et déploiement de techniques de localisation grâce aux outils EDA 
\item Management des outils de navigation
\end{itemize}}

\section{Formation}
\cventry{1995--1998}{\'Ecole d'ingénieur}{Supélec}{Rennes}{\textit{Options micro-électronique}}{}
\cventry{1991--1995}{\'Ecole préparatoire TA}{Lycée Léonce Vieljeux}{La Rochelle}{}{}
\cventry{1989--1991}{BAC F2 électronique}{Lycée Louis Armand}{Poitiers}{}{}
\cventry{1987--1989}{BEP électronique}{Lycée Edouard Branly}{Châtellerault}{}{}

\section{Langues}
\cvitemwithcomment{Anglais}{Lu, parlé, écrit}{}

\section{Compétences informatiques}
\cvdoubleitem{Simulateur}{NCsim, VCS, Questa, Eldo}{Verification}{Specman, Certitude, vManager}
\cvdoubleitem{ATPG}{TetraMax}{LVS}{Calibre}
\cvdoubleitem{Languages}{Perl, Tcl, Java, C/C++, SQL}{HDL}{VHDL, Verilog, e, SystemC}
\cvdoubleitem{BI}{BIRT, QlikView}{Autres}{Git, Subversion, Jenkins}

\section{Conférences}
\cventry{2012}{DAC}{\fontsize{10}{11}\selectfont "Integration of Enterprise Manager SQL database in Verification-Cockpit"}{Poster}{}{}
\cventry{2009}{DVCON}{\fontsize{10}{11}\selectfont "Unified Formal and Dynamic Verification Closure: Can Mutations Bridge the Gap?"}{Papier}{}{}

\section{Centres d'intérêts}
\cvdoubleitem{Culture}{Cinéma, Séries TV, Musique}{Informatique}{Jeux vidéo}
\cvitem{Sport}{Vélo et Badminton}

\end{document}
